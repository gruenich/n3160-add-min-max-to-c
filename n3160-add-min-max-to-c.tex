\documentclass[a4paper,10pt]{scrartcl}
\usepackage[utf8]{inputenc}
\usepackage{amsmath}
\usepackage{listings}
\usepackage{url}
\usepackage{xcolor}
\usepackage{hyperref}

%opening
\title{N3160: Add min, max for integers to C}
\author{Christoph Grüninger\footnote{Unaffiliated, foss@grueninger.de}}
\date{2023-08-06}

\begin{document}

\lstset{language=C,basicstyle=\ttfamily}

\maketitle

\noindent
Document: N3160\\
Audience: WG14\\
Proposal category: New feature\\
Document hosted at \url{https://github.com/gruenich/n3160-add-min-max-to-c}

\section{Summary of proposed changes}

Add the functions \lstinline{min} and \lstinline{max} for integers to the C programming language.

\section{Rationale}

For floats and doubles, \lstinline{fmin} and \lstinline{fmax} exists, but not for integer types. Other
programming languages provide them including C++ by \lstinline{std::min} and \lstinline{std::max}. Several
C libraries provide them as macros, too, e.g., GNU C library \cite{GLibc}, Linux \cite{Linux},
SuperLU \cite{SuperLU}, Cairo \cite{Cairo}.
Stack Overflow has a question where to find C's \lstinline{min}/\lstinline{max} function with more
then 400 upvotes \cite{Stackoverflow}, indicating that many C programmers expect this function to be
part of the C language.

Some may argue these two functions are trivial to implement. C provides other functions that would be trivial to implement like \lstinline{abs} and \lstinline{fabs}. On the other hand, N1292 \cite{N1292} discusses adding niche functions like Legendre polynomials, irregular modified cylindrical Bessel functions, and Gauß' beta function.

\section{Proposal}
Add the $\operatorname{min}$ and $\operatorname{max}$ functions for integers to \lstinline{stdlib.h}.

The formal definitions of the two functions are
\begin{equation*}
 \operatorname{min}(a, b) := \begin{cases}b & a > b\\ a & \text{else}\end{cases}
\end{equation*}
and
\begin{equation*}
 \operatorname{max}(a, b) := \begin{cases}b & b > a\\ a & \text{else}\end{cases}.
\end{equation*}
The following functions should be added to \lstinline{stdlib.h} implementing above definition:
\begin{lstlisting}
 int min (int a, int b);
 int max (int a, int b);
 long int lmin (long int a, long int b);
 long int lmax (long int a, long int b);
 long long int llmin (long long int a, long long int b);
 long long int llmax (long long int a, long long int b);
\end{lstlisting}


\section{Proposed wording}
There is no proposed wording yet. Once this idea gains some support in WG14, we will add a concrete proposal.

\section{Looking for help}
If someone is interested to chip in with experience, WG14 seniority, or as a champion to present the proposal in a committee meeting, we welcome additional authors to this proposal.

\begin{thebibliography}{9}
\bibitem{GLibc}
GNU C Library source code containing macros for MIN and MAX, \url{https://sourceware.org/git/?p=glibc.git;a=blob;f=misc/sys/param.h;hb=HEAD}.

\bibitem{Linux}
Linux source code containing macros for MIN and MAX, \url{https://github.com/torvalds/linux/blob/master/include/linux/minmax.h}.

\bibitem{SuperLU}
SuperLU source code containing macros for MIN and MAX,\\ \url{https://github.com/xiaoyeli/superlu/blob/master/SRC/slu_util.h}.

\bibitem{Cairo}
Cairo source code containing macros for MIN and MAX,\\ \url{https://cgit.freedesktop.org/cairo/tree/perf/cairo-perf.h}.

\bibitem{Stackoverflow}
Stackoverflow question \emph{MIN and MAX in C},\\ \url{https://stackoverflow.com/questions/3437404/min-and-max-in-c}.

\bibitem{N1292}
ISO/IEC JTC 1/SC 22/WG14 N1292 \emph{Extensions to the C Library, to Support Mathematical Special Functions}, 2008.
\end{thebibliography}

\end{document}
